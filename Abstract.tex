Graph network machine learning structures may become effective tools in  computational molecular dynamics (MD). Already, there have been techniques developed using topological constraints based on features such as atomic flexibility and internal stress of materials to predict their chemical reactivity and composition \cite{bauchy}. This project will explore and outline a novel approach to predicting fracture nucleation and fracture propagation in silicate glasses using a graph network to study local structure of silicate atoms. 

Molecular dynamics modeling at the atomic scale has been conventional method of studying fracture behavior. \textbf{It is difficult to isolate the role chemical and mechanical stresses have on fractures since both mechanisms interact. } To bypass this complication, researchers at Sandia National Laboratories have been exploring new methods to supplement and potentially replace the current methods to improve computational efficiency as well as prediction accuracy. 

The 2019-2020 Clinic project will predict fracture nucleation and propagation by using machine learning techniques on a proper graph network representation of silicate glass. We will identify features in local structures of these graphs that may suggest fracture nucleation.