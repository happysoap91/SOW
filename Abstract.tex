Brittle materials are prevalent in industrial applications. A fundamental engineering challenge is predicting where failure can occur in such materials. This involves determining where fracture nucleation will occur, as well as how these fractures propagate.

Researchers at Sandia National Laboratories have successfully used molecular dynamics (MD) simulations to model the effects leading to fracture nucleation and growth in silica-based glasses, given an atomic-level description of the material. However, these simulations are too computationally intensive to allow a systematic study of the relation between atomic structure and fracture behavior.

This Clinic project develops a novel approach to predicting fracture growth in silicate glasses, using a graph theoretic description of the material and machine learning. By training a supervised learning algorithm on MD simulation data, we aim to develop a surrogate model that rapidly and accurately predicts where fractures will emerge under multiple environmental conditions. The model will provide new insight into how local structure, beyond a simple ``weakest link'' description, leads to fracture and failure.

%$\indent$ Predicting where failure occurs in a brittle material plays a critical role in many engineering topics. These brittle materials are prevalent in many industrial applications. The primary challenge in these studies is not only locating the nucleation of a fracture but making realistic lifespan prediction on how fractures propagate. 
%
% Currently, molecular dynamic simulations are the only way to incorporate the effects of chemistry in fracture modeling. Researchers at Sandia National Laboratories have been exploring new methods to supplement the current methods to improve computational efficiency as well as prediction accuracy. 
%
% The Clinic team aims to incorporate a graph theoretic approach based on labeled MD simulated silica-based glasses in order to accurately and efficiently predict fracture nucleation and propagation. Environmental and chemical considerations will be taken into account and supervised machine learning methods will be used to create a surrogate model that can be validated on existing simulation results. 


 
 
 %Graph network machine learning structures may become effective tools in  computational molecular dynamics %(MD). Already, there have been techniques developed using topological constraints based on features such as %atomic flexibility and internal stress of materials to predict their chemical reactivity and composition %\cite{bauchy}. This project will explore and outline a novel approach to predicting fracture nucleation and %fracture propagation in silicate glasses using a graph network to study local structure of silicate atoms.
 
 %The clinic team will address the problem of predicting fracture propagation and nucleation in MD simulated
%silica-based glasses in environmental conditions with local atomic structure. A graph theoretical approach
%%will be used to create a surrogate model, based on supervised learning techniques that are trained on MD
%data, to provide a rapid prediction of the fully atomistic simulation. These predictions will then be correlated
%back to atomic structure, and validated on existing large-scale atomic MD simulation results.