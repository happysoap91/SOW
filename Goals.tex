$\indent$ In this project, our ultimate goal is using machine learning methods to learn from Molecular Dynamics simulations, so that the trained machine learning models can produce rapid predictions of simulation results without the extensive computation that simulations would take. To realize the ultimate goal, we separate it into two sub-goals. One of the sub-goals is to predict fracture nucleation and the other one is to predict fracture propagation. The nucleation event is where fractures will emerge and must be predicted in a certain region and fracture propagation is a continuing process of atoms being included in the fracture region, also known as the process zone. And we will use two different models to address these problems respectively. 



\begin{itemize}
\item \textbf{Goal 1:} \emph{Predicting Fracture Nucleation Events} 

A nucleation event is where a fracture emerges from the structure without any previous fracture existence. We assume that a nucleation event is primarily influenced by the local structure at the atomic level of the material. And an ensemble of random forest model is built to test our hypothesis, by identifying these latent structures from the data. Detailed explanation is presented in the method section.
\end{itemize}

\begin{itemize}
\item \textbf{Goal 2:} \emph{Predicting Fracture Propagation} 

Once a fracture nucleates in the material, it propagates along the existing fracture, as stress keeps being applied uni-axially. Our goal is to identify which atoms will be positive as the fracture propagates. In this case, the input will be a time series of features. And the Long-short Term Memory(LSTM) neural networks is used to capture the dynamic process of the fracture propagation. The reason why we choose LSTM over RNN is that it's a more flexible model and can control how much of a memory we would like the model to contain when making predictions.

\end{itemize}
