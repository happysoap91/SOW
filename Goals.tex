In this project, our ultimate goal is using machine learning methods to learn from Molecular Dynamics simulations, so that the trained machine learning models can produce rapid predictions of simulation results without the extensive computation that simulations would take. To realize the ultimate goal, we separate it into two sub-goals. One of the sub-goals is to predict fracture nucleation and the other one is to predict fracture propagation. The nucleation event is where fractures will emerge and must be predicted in a certain region and fracture propagation is a continuing process of atoms being included in the fracture region, also known as the process zone. And we will use two different models to address these problems respectively. 



\begin{itemize}
\item \textbf{Goal 1:} \emph{Predicting Fracture Nucleation Events} 
\begin{itemize} 
    \item Definition of fracture nucleation: Nucleation event is the creation of void in the material. As stress is applied uni-axially, bonds break and form among atoms. Correspondingly, some atoms have growing void among themselves and others are squished into a more dense region. And fracture nucleates alongside the void occurrence. 
    
    As a result, we define that whether an atom/a node is part of a nucleation event, using a local density measure, which is the Voronoi cell volume $v_i$, around a certain atom $i$. It is a volume measure because we are measuring the volume around the atom in a 3D graph. 
    So, if $v_i$ exceeds a certain threshold, then atom $i$ is defined as part of a nucleation event. In this project, we set the threshold to be 3 standard deviation from the mean value, $\mu_v$, at the current time step.
    \textbf{1. Could insert a figure here demonstrating 3D Voronoi volume}
    \textbf{2. Could insert 3 figures demonstrating why we choose 3sd as our threshold}
    \[
    nuc_v = \frac{(v_i - \mu_v)}{\sigma_v}
    \]  
    \item Contrary to a fracture propagation event, where there is existing void, and fracture just keeps growing along the existing fracture, a nucleation event has no existing fracture to propagates along. As a result, it is different to predict nucleation event. We assume that a nucleation event is primarily influenced by the local structure at the atomic level of the material. And a Recurrent Neural Network model is built to identify these latent structures. Detailed explanation is presented in the method section.
    
\end{itemize}
\end{itemize}


\begin{itemize}
\item \textbf{Goal 2:} \emph{Predicting Fracture Propagation} 
\begin{itemize} 
    \item Definition of fracture in a dynamic setting:
    
    \item Method: We are going to use a derivation of recurrent neural network(RNN), the long-short term memory networks(LSTM) to predict the propagation of an existing interstices. The reason why we choose LSTM over RNN is that it's a more flexible model and can control how much of a memory we would like the model to contain when making predictions.
    
    
    
    
    
\end{itemize}



\end{itemize}
